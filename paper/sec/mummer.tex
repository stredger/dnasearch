

\section{Mummer}

I modified the python script to use MUMmer instead of Bowtie2 as MUMmer showed some matching sequence even if no matches were present in the overall alignment. MUMmer reports sequences larger than a given minimum size, in this case $20$ bp, so even in highly conserved sequence some minimal matches are likely to be reported. The job now required the reference sequence to not be in an index so the crawler was used to grab the GRCh38 reference sequence from the NCBI database into the filesystem. To begin the alignments each node grabs a full copy of the chromosome reference set and stores it locally. Then like in the Bowtie2 experiment each node is responsible for a partition of the viral genomes. However unlike the Bowtie2 experiment each genome must be aligned to each chromosome separately. The reference set of chromosomes contains chromosomes $1$ - $22$, an X and Y, as well as the mitochondrial dna (MT), and some unmapped sequence (UT). This means that each genome is aligned against $26$ chromosome sequences. Once an alignment has completed the output file is uploaded back into the sage filesystem. The job was started and from observation it takes approximately $2$ minutes to perform an alignment. Unfortunately there are $5534$ viral sequences and $26$ chromosomes, so about $143884$ alignments must be made. Dividing the work over three machines at $2$ mins per alignment means the job will take approximately $95922$ minutes, about $67$ days! In order to get the alignments within a day I would need around $200$ machines each with over $4$ GB of ram. I decided to not complete the job as the time commitment on the GEE machines was too large.