

\section{Introduction}

Since the 1970’s when viral DNA, or fragments that resemble viral DNA were found to exist in the human genome, researches have been investigating how it got there, and what role if any it plays in development. Such viral elements are known as endogenous viral elements and the most well known in humans come from a family of viruses called retroviruses. What makes retroviruses unique is that they essentially incorporate their own DNA into that of its host through a process called reverse transcription. The most well known human infecting retrovirus is the human immunodeficiency virus (HIV) which is classified as an exogenous retrovirus. Retroviral elements that are vertically transmitted (passed to siblings) in human DNA are known as human endogenous retroviral elements (HERVs) and it is estimated that $4$-$8$\% of the human genome is composed of HERVs \cite{eves}. While these HERVs are the most common viral elements found in our genome, other viral elements have been shown to exist in animal genomes and are generally called endogenous viral elements (EVEs). Most EVEs are quite modified from the origin fragment, but this leads to an interesting question on how easy it is to find EVEs in our genome? Can we find any suspect sequence with a simple distributed system?